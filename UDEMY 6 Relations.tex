

%-------------------------------------------------------------------------------- %


%------------------------------------------------------ 
\subsection*{Relations}
Let S be a setr A relation ’R, on S is a rule which compares any two elements
23, $y \in S$ and tells us either that 92 is related to y or that rc is not related to y. 

We write $x\mathcal{R}y$ to mean “x is related to y under the relation $\mathcal{R}$". We are already familiar with many examples of relations defined in society: for example we could let S be
the set of all people in London and say that two people $x$ and $y$ are related if $x$ is a
parent of $y$. We have also already seen several examples of relations in mathematics,
for example "::", "<", and "$\leq$" are all relations on the set. ofiritegers.

A different example is the following:

%
%------------------------------------------------------ %


\textbf{Example 1.3} Let $S = \{2,3\}$. We define a relation 72 on S by saying that z is
related to y if $|x — y| = 1$; for all any 6 S. Thus 172,2 but 1 is not related to 3.

\subsubsection*{Using digraphs to illustrate relations}
Given a relation $\mathcal{R}$ on a set S; we can model R by defining the digraph D with
V(D] = S in which, for any i-wo vertices cr and yr there is an arc in D from z to y if
and only if x'/'ly. We shall call D the relationship digraph corresponding to 72.
%------------------------------------------ %


\textbf{Example 1.4} Figure 1.2 gives the relationship digraph corresponding to the rela-
tion defined in Example 1.3.

Conversely, given a digraph D. we can define a relation $\mathcal{R}$ on the set S : i/'(D)
by saying $x\mathcal{R}y$ if and only if there is an arc in D from ar to y
%------------------------------------------ %


\textbf{Example 1.5} The digraph given in Figure 1,1 defines the relation 72 on thc sor
S = {viwg, M3} given by z=1’Rvg, vglivg, ug’I2vi, ¤3’!?.v3, and vgiiv;.

\subsection*{Equivalence Relations}
We can see from Example 1.5 that the definition ofa relation on at ser may be rather
abitrary. Many relations which occur in practical situations however have a more
``\textit{regular structure}".

%------------------------------------------ %
\noindent \textbf{Definition 1.6} Let $\mathcal{R}$ be a relation defined on a set $S$. We say that $\mathcal{R}$ is:
\begin{itemize}
\item reflexive if for all $x \in S$, we have $x\mathcal{R}x$.
\item symmetric if for all $x, y \in S$ such that $x\mathcal{R}y$, we have $y\mathcal{R}x$.
\item transitive if for all $x, y, z \in S$ such that $x\mathcal{R}y$ and $y\mathcal{R}z$, we have $x\mathcal{R}z$.
\end{itemize}
%------------------------------------------ %
In terms of the relationship digraph D 0f'R, it can be seen that:
\begin{itemize}
\item $\mathcal{R}$ is reflexive if all vertices of D are in a directed loop.
\item $\mathcal{R}$ is symmetric if all arcs my in D are in a directed cycle of length two.
\item $\mathcal{R}$ is transitive if for all directed paths of length two P : xyz of D, we have
\end{itemize}
an are xz; and for all directed cycles of length two C : mym 0f D, we have a
loop 1::: and a loop yy.

Using Figure 1.2 we deduce that the relation defined in Example 1.3 is reflexive
and symmetric but not transitive. 
Similarly, using Figure 1.1 we deduce that the
relation deiined in Example 1.5 is not reliexive, symmetric or transitive.
%------------------------------------------ %

\noindent \textbf{Example 1.7} The relation $\leq $ defined on $\mathbb{Z}$ is reflexive and transitive but not
symmetric. The relation $<$ defined on $\mathbb{Z}$ is transitive but not reflexive or symmetric.
\end{document}