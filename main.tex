
% http://www.textbooksonline.tn.nic.in/books/12/std12-bm-em-1.pdf

\large
\begin{itemize}
\item A relation $R$ from a set A to a set B is a subset of the
\textbf{cartesian product} A x B. 
\item Thus R is a set of \textbf{ordered pairs} where
the first element comes from A and the second element comes
from B i.e. $(a, b)$
\end{itemize}

\large
\begin{itemize}

\item  If $(a, b) \in R$ we say that $a$ is related to $b$ and write $aRb$.
\item If $(a, b) \notin R$, we say that $a$ is not related to $b$ and write $aRb$. CHECK
\item If
$R$ is a relation from a set $A$ to itself then we say that ``$R$ is a relation
on $A$".

\item Let $A = \{2, 3, 4, 6\}$ and $B = \{4, 6, 9\}$
\item Let R be the relation from A to B defined by \textit{\textbf{xRy}} if $x$
divides $y$ exactly.


\item Let $A = \{2, 3, 4, 6\}$ and $B = \{4, 6, 9\}$
\item Let R be the relation from A to B defined by \textit{\textbf{xRy}} if $x$
divides $y$ exactly.
\item  Then
\[R = {(2, 4), (2, 6), (3, 6), (3, 9), (4, 4), (6, 6)}\]
\end{itemize}



\end{document}


\begin{itemize}
\item \textbf{Reflexive}: 
\item \textbf{Symmetric}: 
\item \textbf{Transitive}: 
\item \textbf{Anti-symmetric}: 
\item \textbf{Equivalence Relation}: 
\item \textbf{Partial Order}:
\item \textbf{Order}:
\end{itemize}

%-----------------------------------------------------%
\section*{Question 6}

\subsection*{Part A : Digraphs}

Suppose $A = \{1,2,3,4\}$. Consider the following relation in A

\[ \{  (1,1),(2,2),(2,3),(3,2),(4,2),(4,4)\} \]

Draw the direct graph of $A$. Based on the Digraph of $A$ discuss whether or not a relation that could be depicted by the digraph could be described as the following, justifying your answer.


\begin{itemize}
\item[(i)] Symmetric
\item[(ii)] Reflexive 
\item[(iii)] Transitive
\item[(iv)] Antisymmetric
\end{itemize}
\subsection*{Part B : Relations}
Determine which of the following relations $ x R y$ are reflexive, transitive, symmetric, or antisymmetric on the following - there may be more than one characteristic.  if

\begin{itemize} 
\item[(i)] $x = y$
\item[(ii)] $x < y$
\item[(iii)] $x^2 = y^2$
\item[(iv)] $x \geq y$
\end{itemize}
\subsection*{Part C : Partial Orders}
% http://staff.scem.uws.edu.au/cgi-bin/cgiwrap/zhuhan/dmath/dm_readall.cgi?page=20

Let $A=\{0,1,2\}$ and $R=\{ (0,0),(0,1),(0,2),(1,1), (1,2), (2,2)\}$
and $S=\{(0,0),(1,1),(2,2)\}$ be 2 relations on A. Show that

\begin{itemize}
\item[(i)] R is a partial order relation.
\item[(ii)] S is an equivalence relation.
\end{itemize}



\section{Digraphs and Relatiosn}
% 2007 Q8
Given a flock of chickens, between any two chickens one of them is
dominant. A relation, R, is defined between chicken x and chicken y as xRy if x is
dominant over y. This gives what is known as a pecking order to the flock. Home
Farm has 5 chickens: Amy, Beth, Carol, Daisy and Eve, with the following relations:

\begin{itemize}
\item Amy is dominant over Beth and Carol
\item Beth is dominant over Eve and Carol
\item Carol is dominant over Eve and Daisy
\item Daisy is dominant over Eve, Amy and Beth
\item Eve is dominant over Amy.
\end{itemize}

\newpage
