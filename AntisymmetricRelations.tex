\documentclass[12pt]{article}


\begin{document}


\section*{Antisymmetric Relations}

\begin{itemize}
\item A binary relation $R$ on a set $X$ is \textbf{antisymmetric} if there is no pair of distinct elements of X each of which is related by $R$ to the other. 

\item More formally, $R$ is antisymmetric precisely if for all a and b in $X$ :\\
if $R(a,b)$ and $R(b,a)$, $then a = b$,

\item Intuitively, an antisymmetric relation has no symmetric pairs. Consider the relation: 

\[R = { (0,0) , (0,1) , (1,0) } ,\] this relation is symmetric. 

In the example stated, the pair (1,0) and (0,1) are symmetric, so this violates the antisymmetric condition. An example of a relation that is antisymmetric is $R=\{(0,0),(0,1)\}$.
\end{itemize}
%--------------------------------------------------------%
\end{document}
