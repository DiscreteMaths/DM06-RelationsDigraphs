\documentclass[]{report}

\voffset=-1.5cm
\oddsidemargin=0.0cm
\textwidth = 480pt

\usepackage{framed}
\usepackage{subfiles}
\usepackage{graphics}
\usepackage{newlfont}
\usepackage{eurosym}
\usepackage{amsmath,amsthm,amsfonts}
\usepackage{amsmath}
\usepackage{color}
\usepackage{enumerate}
\usepackage{amssymb}
\usepackage{multicol}
\usepackage[dvipsnames]{xcolor}
\usepackage{graphicx}
\begin{document}
	
	Let R be an equivalence relation defined on a set S and let x 2 S. 
Then the equivalence class of x is the subset of S containing all
elements of S which are related to x. 

%================================================================================ %

	
	\begin{itemize}
	\item We denote this by [x]. Thus
	\[[x] = {y 2 S : yRx}\].A. Let S = {0, 1, 2, 3}. 
	\item Define a relation R1 between the elements of S by
	“x is related to y if the product xy is even”.
	\end{itemize}



%================================================================================ %

	
\section{Exercise B}
Let $S = \{0, 1, 2, 3\}$. 
Define a relation R2 between the elements of S by
\[\mbox{“x is related to y if x − y} \in {0, 3,−3}”.\]

%================================================================================ %

\subsection{Example}	
C. Let S = {{1}, {1, 2}, {1, 2, 3}, {1, 2, 4}}. 
Define a relation R3 between
the elements of S by “X is related to Y if $X \subseteq Y$ ”.

%================================================================================ %
\begin{itemize}

%\begin{emunerate}[(i)]	
\item Draw the relationship digraph.
\item  Determine whether the relation is either reflexive, symmetric, transitive
or anti-symmetric. For the cases when one of these properties
does not hold, justify your answer by giving an example to show that
it does not hold.

\item  Determine which of the relations is an equivalence relation. For the
case when it is an equivalence relation, calculate the distinct equivalence
classes and verify that they give a partition of S.
\item Determine which of the relations is a partial order or an order.
%------------------------------------------------------------%
\end{itemize}
%\end{emunerate}
% Part A
The relation:
\begin{itemize}
\item is not reflexive e.g. 1 is not related to 1;
\item is symmetric;
\item is not transitive e.g. 1 is related to 2, and 2 is related to 3, but 1 is not related to 3;
\item is not anti-symmetric e.g. 1 is related to 2, and 2 is related to 1,
but 1 6= 2.
\end{itemize}


%================================================================================ %

	
(iii) R1 is not an equivalence relation because it is neither reflexive
nor transitive.
(iv) R1 is not a partial order or an order.

%================================================================================ %

	
% Part B
The relation:
\begin{itemize}
    \item is reflexive, symmetric and transitive;
\item is not anti-symmetric e.g. 0 is related to 3, and 3 is related to 0,
but 0 6= 3.
\end{itemize}


%================================================================================ %

\begin{itemize}
    \item (iii) R2 is an equivalence relation. The three equivalence classes are
{0, 3}, {1}, and {2}. These form a partition of {0, 1, 2, 3}.
(iv) R2 is not a partial order or an order as it is not anti-symmetric.


% Part C
\item ii) The relation:
is reflexive, transitive and anti-symmetric;
is not symmetric e.g. {1} is related to {1, 2}, but {1, 2} is not
related to {1}.

\item (iii) R3 is not an equivalence relation as it is not symmetric.

\item (iv) R3 is a partial order but not an order (since {1, 2, 3} and {1, 2, 4}
are not related to each other).
\end{itemize}	

%================================================================================ %

	


%================================================================================ %
\end{document}
