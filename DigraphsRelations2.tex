\documentclass[]{article}
\usepackage{framed}
\usepackage{amsmath}
\usepackage{amssymb}
\usepackage{graphicx}
\usepackage{multicol}
%opening

\begin{document}
\subsection*{Chapter 6: Digraphs and Relations}
%--------------------------------------------- %
\textbf{Example 1.19} Let X be the set of all students registered at a college and Y be
the set or" atl courses being taught at the college. Then the college database keeps
a record of the relationship in which a student z is related to a course y if 1: is
registered for y.
%--------------------------------------------- %
We can model a relation 'R between sets X and Y by a digraph D in much the
same way as we did previously for relations on a. single set: We put l/(D) = X U Y
and draw an arc from a vertex m 5 X to a vertex y E Y if x7?.y.
%---------------- %
Example 1.20 Let $X = \{1, 2,3\}$ and $Y = \{b, c\}$. Define R by l’R.b, 1Rc, 272::, and
3‘I?.c. Then the relationship digraph for 'R is shown in Figure 1.3.
%--------------------------------------------- %
There is at strong similarity between Figure 1.3 and the figures drawn in Volume
1, Chapter 4 to illustrate functions, Indeed we can view a function $f : X \rightarrow Y$ as
being a relation in which each a.- E X is related to a unique element of Y. We write
f(:c) to represent the unique element of Y which is related to m.



